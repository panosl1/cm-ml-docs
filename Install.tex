\documentclass[]{article}

\usepackage{lmodern}

\usepackage{amssymb,amsmath}

\usepackage{ifxetex,ifluatex}

\usepackage{fixltx2e} % provides \textsubscript

\ifnum 0\ifxetex 1\fi\ifluatex 1\fi=0 % if pdftex

  \usepackage[T1]{fontenc}

  \usepackage[utf8]{inputenc}

\else % if luatex or xelatex

  \ifxetex

    \usepackage{mathspec}

  \else

    \usepackage{fontspec}

  \fi

  \defaultfontfeatures{Ligatures=TeX,Scale=MatchLowercase}

\fi

% use upquote if available, for straight quotes in verbatim environments

\IfFileExists{upquote.sty}{\usepackage{upquote}}{}

% use microtype if available

\IfFileExists{microtype.sty}{%

\usepackage{microtype}

\UseMicrotypeSet[protrusion]{basicmath} % disable protrusion for tt fonts

}{}

\usepackage[unicode=true]{hyperref}

\hypersetup{

            pdfborder={0 0 0},

            breaklinks=true}

\urlstyle{same}  % don't use monospace font for urls

\usepackage{longtable,booktabs}

\IfFileExists{parskip.sty}{%

\usepackage{parskip}

}{% else

\setlength{\parindent}{0pt}

\setlength{\parskip}{6pt plus 2pt minus 1pt}

}

\setlength{\emergencystretch}{3em}  % prevent overfull lines

\providecommand{\tightlist}{%

  \setlength{\itemsep}{0pt}\setlength{\parskip}{0pt}}

\setcounter{secnumdepth}{0}

% Redefines (sub)paragraphs to behave more like sections

\ifx\paragraph\undefined\else

\let\oldparagraph\paragraph

\renewcommand{\paragraph}[1]{\oldparagraph{#1}\mbox{}}

\fi

\ifx\subparagraph\undefined\else

\let\oldsubparagraph\subparagraph

\renewcommand{\subparagraph}[1]{\oldsubparagraph{#1}\mbox{}}

\fi



\date{}



\begin{document}



\section{Installing the Machine Learning

RPMs}\label{installing-the-machine-learning-rpms}



As of Bright Cluster Manager 7.3 a number of Machine/Deep Learning

library and framework RPM packages have been included. Bright makes it

faster and easier for organizations to gain actionable insights from

rich, complex data using the latest, state-of-the-art libraries with

minimal effort during installation.



Currently the following RPMs are available:



\begin{longtable}[]{@{}ll@{}}

\toprule

\begin{minipage}[b]{0.17\columnwidth}\raggedright\strut

Package name\strut

\end{minipage} & \begin{minipage}[b]{0.18\columnwidth}\raggedright\strut

Description\strut

\end{minipage}\tabularnewline

\midrule

\endhead

\begin{minipage}[t]{0.17\columnwidth}\raggedright\strut

cm-ml-distdeps\strut

\end{minipage} & \begin{minipage}[t]{0.18\columnwidth}\raggedright\strut

Meta-package containing library dependencies available from the base

Linux distribution repositories as well as the EPEL repository\strut

\end{minipage}\tabularnewline

\begin{minipage}[t]{0.17\columnwidth}\raggedright\strut

cm-ml-pythondeps\strut

\end{minipage} & \begin{minipage}[t]{0.18\columnwidth}\raggedright\strut

Pre-packaged Python dependencies for Bright's RPM packages\strut

\end{minipage}\tabularnewline

\begin{minipage}[t]{0.17\columnwidth}\raggedright\strut

caffe\strut

\end{minipage} & \begin{minipage}[t]{0.18\columnwidth}\raggedright\strut

A deep learning framework made with expression, speed, and modularity in

mind.\strut

\end{minipage}\tabularnewline

\begin{minipage}[t]{0.17\columnwidth}\raggedright\strut

torch7\strut

\end{minipage} & \begin{minipage}[t]{0.18\columnwidth}\raggedright\strut

A library in C++ for developing Open Source speech and machine learning

applications.\strut

\end{minipage}\tabularnewline

\begin{minipage}[t]{0.17\columnwidth}\raggedright\strut

theano\strut

\end{minipage} & \begin{minipage}[t]{0.18\columnwidth}\raggedright\strut

Theano is a Python library that allows you to define, optimize, and

evaluate mathematical expressions involving multi-dimensional arrays

efficiently.\strut

\end{minipage}\tabularnewline

\begin{minipage}[t]{0.17\columnwidth}\raggedright\strut

tensorflow\strut

\end{minipage} & \begin{minipage}[t]{0.18\columnwidth}\raggedright\strut

TensorFlow is an Open Source Software Library for Machine

Intelligence\strut

\end{minipage}\tabularnewline

\begin{minipage}[t]{0.17\columnwidth}\raggedright\strut

cudnn\strut

\end{minipage} & \begin{minipage}[t]{0.18\columnwidth}\raggedright\strut

A GPU-accelerated library of primitives for deep neural networks.\strut

\end{minipage}\tabularnewline

\begin{minipage}[t]{0.17\columnwidth}\raggedright\strut

cub\strut

\end{minipage} & \begin{minipage}[t]{0.18\columnwidth}\raggedright\strut

Reusable software components for CUDA\strut

\end{minipage}\tabularnewline

\begin{minipage}[t]{0.17\columnwidth}\raggedright\strut

nccl\strut

\end{minipage} & \begin{minipage}[t]{0.18\columnwidth}\raggedright\strut

(pronounced ``Nickel'') A stand-alone library of standard collective

communication routines, such as all-gather, reduce, broadcast, etc.,

that have been optimized to achieve high bandwidth over PCIe.\strut

\end{minipage}\tabularnewline

\begin{minipage}[t]{0.17\columnwidth}\raggedright\strut

mlpython\strut

\end{minipage} & \begin{minipage}[t]{0.18\columnwidth}\raggedright\strut

MLPython is a library for organizing machine learning research.\strut

\end{minipage}\tabularnewline

\begin{minipage}[t]{0.17\columnwidth}\raggedright\strut

DIGITS\strut

\end{minipage} & \begin{minipage}[t]{0.18\columnwidth}\raggedright\strut

A web frontend to Caffe and Torch developed by NVIDIA\strut

\end{minipage}\tabularnewline

\bottomrule

\end{longtable}



The following package will be added in the future:



\begin{longtable}[]{@{}ll@{}}

\toprule

\begin{minipage}[b]{0.17\columnwidth}\raggedright\strut

Package name\strut

\end{minipage} & \begin{minipage}[b]{0.18\columnwidth}\raggedright\strut

Description\strut

\end{minipage}\tabularnewline

\midrule

\endhead

\begin{minipage}[t]{0.17\columnwidth}\raggedright\strut

caffeonspark\strut

\end{minipage} & \begin{minipage}[t]{0.18\columnwidth}\raggedright\strut

CaffeOnSpark brings deep learning to Hadoop and Spark clusters. By

combining salient features from deep learning framework Caffe and

big-data frameworks Apache Spark and Apache Hadoop, CaffeOnSpark enables

distributed deep learning on a cluster of GPU and CPU servers.\strut

\end{minipage}\tabularnewline

\begin{minipage}[t]{0.17\columnwidth}\raggedright\strut

cntk\strut

\end{minipage} & \begin{minipage}[t]{0.18\columnwidth}\raggedright\strut

The Computational Network Toolkit by Microsoft Research, is a unified

deep-learning toolkit\strut

\end{minipage}\tabularnewline

\begin{minipage}[t]{0.17\columnwidth}\raggedright\strut

bidmach\strut

\end{minipage} & \begin{minipage}[t]{0.18\columnwidth}\raggedright\strut

BIDMach is a very fast machine learning library.\strut

\end{minipage}\tabularnewline

\bottomrule

\end{longtable}



\subsection{Requirements}\label{requirements}



The following requirements must be met before installing the Machine

Learning packages



\begin{itemize}

\tightlist

\item

  RHEL, CentOS or Scientic Linux 7.x

\item

  Access to the Linux distribution's online YUM repositories as well the

  EPEL repository

\item

  2 GB of free space for the RPMs that are installed on the headnode and

  an additional 400 MB for each software image that will be used for

  running Machine Learning pipelines

\item

  (Recommended) Maxwell or more recent NVIDIA GPUs with compute

  capability 3.5 or later

\end{itemize}



\subsection{Installation}\label{installation}



\subsubsection{For the compute nodes}\label{for-the-compute-nodes}



The only RPM that needs to be installed on the headnode is

cm-ml-distdeps. This is is a meta-package the instructs YUM to install

the necessary system libraries as well as the development packages

e.g.~blas-devel. If for example the name of the software image is

gpu-image, you can install the RPM by running the following command:



\begin{verbatim}



yum install --installroot=/cm/images/gpu-image cm-ml-distdeps

\end{verbatim}



You will need to repeat the above command for all software images that

will be used to run Machine Learning applications.



\subsubsection{For the headnode}\label{for-the-headnode}



The Bright Cluster Manager Machine learning packages have proper RPM

dependencies defined. This means that the cluster administrator does not

need to spend time figuring out what needs to be installed.



For example, if the administrator wants to install NVIDIA DIGITS, all

that has to be done is to run the command:



\begin{verbatim}



yum install digits

yum install <name of desired package>

\end{verbatim}



on the headnode. YUM will automatically install cm-ml-pythondeps,

cm-ml-distdeps, cudnn, caffe, torch and cuda75-toolkit as dependencies.

These RPMs get installed in the /cm/shared directory, which is exported

over NFS and therefore are available to all the compute nodes. This

allows the installation of the Machine Learning libraries within

minutes, instead of days that it typically takes to build and install

all the necessary dependencies.



\subsubsection{Friendly to developers}\label{friendly-to-developers}



Developers that work on extended the aforementioned libraries will not

want to use the pre-packages RPMs. For this use-case, Bright can help

minimize the time spent to get started.



By installing the cm-ml-pythondeps, cm-ml-distdeps (on the headnode and

compute nodes) and cudnn RPM packages you can get ready for Machine

Learning development within minutes and spend time on the interesting

application at hand, instead of the copious procedure of satisfying

dependencies.



\subsection{Using the Machine Learning

packages}\label{using-the-machine-learning-packages}



Bright provides environment module definitions for all the Machine

Learning packages. The environment module files are also compatible

with Lmod that was introduced in 7.3.



The Machine Learning environment modules automatically load additional

environment modules as dependencies.



For example, loading the DIGITS module with:



\begin{verbatim}



module load shared digits

\end{verbatim}



will automatically load additional modules such cudnn, openblas,

hdf5\_18 etc. that are necessary in order to use DIGITS:



\begin{verbatim}

[root@nwtr-centos7 ~]# module list

Currently Loaded Modulefiles:

  1) shared

\end{verbatim}



\begin{verbatim}



[root@nwtr-centos7 ~]# module load shared digits

[root@nwtr-centos7 ~]# module list

Currently Loaded Modulefiles:

  1) shared                    3) cudnn/5.0                 5) cuda75/toolkit/7.5.18     7) caffe/0.16                9) digits/4.0

  2) cm-ml-pythondeps/1.10.0   4) openblas/dynamic/0.2.18   6) hdf5_18/1.8.17            8) torch7/7.0

[root@nwtr-centos7 ~]#

\end{verbatim}



This is archived via the module definition files:



\begin{verbatim}



[root@nwtr-centos7 ~]# module show digits

-------------------------------------------------------------------

/cm/shared/modulefiles/digits/4.0:



module-whatis    adds nVidia Deep Neural Network Library to your environment variables

module           load caffe

module           load torch7

module           load cm-ml-pythondeps

module           load cudnn

module           load openblas

module           load cuda75/toolkit

module           load hdf5_18

prepend-path     PATH /cm/shared/apps/digits/4.0/

-------------------------------------------------------------------



[root@nwtr-centos7 ~]#

\end{verbatim}



\subsection{Further reading}\label{further-reading}



Additional information about the usage of each individual framework can

be found in the user manual.



\end{document}

